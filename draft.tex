\documentclass{book}
\usepackage[utf8]{inputenc}    
\usepackage[T1]{fontenc}
\usepackage[francais]{babel} 
\usepackage[Lenny]{fncychap}    
      

\title{Lock}
\author{Hoda \bsc{DABONNE}}
\date{3 janvier 2021}
\begin{document}
 
\maketitle
  

\chapter{ Context et Justification }
    
\section{Naissance du projet}
Ce projet est nee de nos experiences personnel et celles de nos proches 
(parents amis et colleges).
Nous avons tous ete un jour victime, temoin ou nous avons entendus parler de vole de motos,
de voiture ou autre engins roulant avec imatriculation et/ou numero de serie et/ou numero
moteur. Le context du terrorisme dans le G5 Sahel apport aussi son lot de difficultes. 
L'insecurite de plus en plus grandissant et les difficultes 
liees aux moyens limites des organisme en charge de notre securite rendent difficile le
travail de nos braves agents de securite. C'est dans ce context que nous avons eux l'idee du projet Lock.

\section{Clarfication de l'idee}
Lock est un un projet logiciel qui vise plusieurs objectifs et finalites
\subsection{Objectifs}
Lock se veut novateur dans la gestion de la securite des engins. Pour ce faire, nous embitionons les objectifs suivants:
\begin{itemize}
    \item permetre aux usager de suivre les itineraires de leur engins;
    \item permetre aux services de securite de pouvoir identifier a temps reel un engin dans le cadre d'une infraction ou d'une recherche;
    \item avoir une base de donnees commune pour les engins;
    \item reduire la lenteur administrative dans les services de securites en ceux qui concerne les retraits des engins implique dans 
    les infractions
    \item doter les engins d'un identifiant unique 
\end{itemize}

\subsection{Fonctionalite}

\subsection{Finalite}
A terme, Lock doit permettre aux usager de disposer d'un systeme securise pour la protection de leur engins. Il doit aussi faciliter 
la gestion des parking et doter les services de securites d'une plateforme unique pour la gestion des cas de perte des 

\section{Eat des lieux}
A notre connaissance, il n'existe aucun projet logiciel qui vise les meme finalites que Lock. 

\chapter{ Budget previsionnel }

\chapter{ Plan d'action }
\end{document}

